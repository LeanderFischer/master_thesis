
Die Beobachtung, dass Neutrinos ihr Lepton-Flavor nach Ausbreitung durch den Weltraum verändern können, ist ein Beweis für ihre von Null verschiedenen Massen und damit ein Indiz f\"ur neue Physik außerhalb des Standardmodells.
IceCube ist ein quadratkilometer großer Cherenkov Neutrino Detektor, der sich unterhalb des geografischen Südpols im antarktischen Gletschereis befindet.
DeepCore, das dichter instrumentierte Subarray von IceCube, ist in der Lage Neutrinos bis herab zu Energien von einigen GeV zu detektieren.

Diese Arbeit steht in enger Verbindung mit der Messung des Verschwindens von atmosphärischen Myon-Neutrinos als einer der möglichen Detektionskanäle für Neutrinooszillationen.
Die Identifizierung des Lepton-Flavor der gemessenen Neutrinos ist ein unerlässlicher Bestandteil um die Neutrinooszillations-Parameter $\Delta \rm{m}^{2}_{32}$ und $\theta_{23}$ zu bestimmen.
Der Kern dieser Arbeit ist die Entwicklung eines neuartigen Verfahrens zur Unterscheidung von Lichtspuren, die von Myon-Neutrinos in geladenen Strömen der schwachen Wechselwirkung erzeugt werden, und Lichtkaskaden, die sowohl von Elektron- und Tau-Neutrinos in geladenen Strömen als auch in allen neutralen Strömen der schwachen Wechselwirkung erzeugt werden.
Das Verfahren verwendet eine mehrdimensionale Methode des Maschinellen Lernens um die Trennung der Event-Klassen gegenüber traditionellen, eindimensionalen Techniken zu verbessern.
Die Anwendung dieser Methode in der Bestimmung der Neutrinooszillations-Parameter führt zu einer Verbesserung der Genauigkeit von $\Delta\rm{m}^{2}_{32}$ um 13,0\,\% und $\sin^{2}(\theta_{23})$ um 7,2\,\%.
