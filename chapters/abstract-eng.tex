
Neutrino oscillations, the phenomenon that neutrinos can change their flavor after propagation through space, is proof of their non-zero masses and, therefore, a sign of new physics beyond the Standard Model.
IceCube is a cubic kilometer Cherenkov neutrino detector buried in the Antarctic glacial ice at the geographic South Pole.
DeepCore is a more densely instrumented sub-array located at the center of IceCube.
It can detect neutrinos down to energies as low as a few GeV.

This work is closely related to measurements of atmospheric muon neutrino disappearance as one of the possible detection channels of neutrino oscillations.
Identifying the flavor of detected neutrinos is essential to determine the neutrino oscillation parameters $\Delta \rm{m}^{2}_{32}$ and $\theta_{23}$.
The core of this thesis is the development of a novel method to distinguish tracks, caused by muon neutrino charged-current interactions, from cascades, caused by both neutral-current interactions of all flavors and charged-current interactions of electron and tau neutrinos.
The method utilizes a Gradient Boosting Machine to enhance the separation between these event classes over the traditional, univariate techniques.
Applying this method to DeepCore data leads to an improvement in the sensitivities to $\Delta\rm{m}^{2}_{32}$ and $\sin^{2}(\theta_{23})$ of 13.0\,\% and 7.2\,\%, respectively.
